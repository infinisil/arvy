\documentclass[12pt,hyperref={pdfpagelabels=false},usenames,dvipsnames]{beamer}
\let\Tiny=\tiny

\usepackage{xspace}
\usepackage{xmpmulti}
\usepackage{array}
\usepackage[absolute,overlay]{textpos}
\usepackage{forloop}
\usepackage{pgfplots}
\usepackage{subcaption}
\usepackage{tikz}
\usepackage{numprint}
\usetikzlibrary{pgfplots.groupplots}

\usetikzlibrary{positioning}
\usetikzlibrary{arrows.meta}

%\usecolortheme{rose}

\setbeamerfont{frametitle}{series=\bfseries}
\setbeamercolor*{title}{fg=white}
\setbeamercolor*{author}{fg=white}
\setbeamercolor*{institute}{fg=white}
\setbeamercolor*{date}{fg=white}
\setbeamerfont{title}{series=\bfseries,size=\huge}
% \setbeamercolor{frametitle}{}
\setbeamerfont{author}{series=\bfseries}


\usetheme{Pittsburgh}
\setbeamertemplate{navigation symbols}{}

\title{Arvy Heuristics for Distributed Mutual Exclusion}
\author{Silvan Mosberger}
\institute{ETH Zurich -- Distributed Computing Group -- www.disco.ethz.ch}

\begin{document}

{
\usebackgroundtemplate{\includegraphics[width=\paperwidth]{figures/bg}}
\begin{frame}
\begin{textblock*}{\paperwidth}[0,0](0cm,1cm)
        \begin{center}
                \usebeamercolor[fg]{title}
                \textbf{\huge \inserttitle}
        \end{center}
        \usebeamercolor[fg]{normal text}
\end{textblock*}
\begin{textblock*}{\paperwidth}[0,0](-0.5cm,7.4cm)
        \flushright
        \color{white}
        \itshape \insertauthor
        \usebeamercolor[fg]{normal text}
\end{textblock*}
\begin{textblock*}{\paperwidth}[0,1](0.2cm,9.4cm)
        \flushleft
        \usebeamercolor[fg]{institute}
        \tiny \itshape \insertinstitute
        \usebeamercolor[fg]{normal text}
\end{textblock*}
\end{frame}
}

% Explain the problem

\begin{frame}{Distributed Mutual Exclusion}

The guarantee that only at most one node in a network can have exclusive access to a shared resource

\begin{block}{Examples}
\begin{itemize}
\item A distributed counter where every node can increase it
\item Shared memory in a multiprocessor machine
\end{itemize}

\end{block}

\end{frame}

\tikzset{
  arvy-expl/.style =
    { v/.style = { circle, draw } % normal graph vertices
    , r/.style = { v, gr } % root nodes
    , rd/.style = { red!70!black }
    , bl/.style = { blue!70!black }
    , gr/.style = { green!50!black }
    , q/.style = { v, rd } % currently making request
    , e/.style = { draw=black, -> }
    , re/.style = { e, dashed, bl }
    , cand/.style = { dashed, black!70!white, ->, >={Stealth[scale=1]} }
    , weight/.style = { dotted, black!30!white }
    , >={Stealth[scale=2]}
    , scale = 1.0
    , auto
    }
}

\begin{frame}{Arrow Algorithm}
\begin{block}{Idea}
Maintain a rooted spanning tree with every node (transitively) pointing towards the node holding the token.
\end{block}
\end{frame}

\begin{frame}{Arrow Algorithm}
\centering

\begin{tikzpicture}[arvy-expl]
\node[v] (1) at (0,5) {1};
\node[q] (2) at (7,5) {2};
\node[rd, above=3pt of 2] {\footnotesize needs resource};
\node[v] (3) at (3,4) {3};
\node[v] (4) at (2,1) {4};
\node[r] (5) at (6,0) {5};
\node[gr, above=0 of 5] {\footnotesize has resource};
\draw[e] (2) -- (3);
\draw[e] (3) -- (4);
\draw[e] (4) -- (5);
\draw[e] (1) -- (4);
\path (5) edge [loop below] (4);
\end{tikzpicture}

\end{frame}

\begin{frame}{Arrow Algorithm}
\centering

\begin{tikzpicture}[arvy-expl]
\node[v] (1) at (0,5) {1};
\node[q] (2) at (7,5) {2};
\node[rd, above=3pt of 2, color=white] {\footnotesize needs resource};
\node[v] (3) at (3,4) {3};
\node[v] (4) at (2,1) {4};
\node[r] (5) at (6,0) {5};
\draw[re] (2) --node[below right=3pt and -20pt]{\footnotesize request from 2}  (3);
\draw[e] (3) -- (4);
\draw[e] (4) -- (5);
\draw[e] (1) -- (4);
\path (5) edge [loop below] (4);
\path (2) edge [loop above] (2);
\end{tikzpicture}
\end{frame}

\begin{frame}{Arrow Algorithm}
\centering
\begin{tikzpicture}[arvy-expl]
\node[v] (1) at (0,5) {1};
\node[q] (2) at (7,5) {2};
\node[rd, above=3pt of 2, color=white] {\footnotesize needs resource};
\node[v] (3) at (3,4) {3};
\node[v] (4) at (2,1) {4};
\node[r] (5) at (6,0) {5};
\draw[e] (3) -- (2);
\draw[re] (3) --node[right]{\footnotesize request from 2}  (4);
\draw[e] (4) -- (5);
\draw[e] (1) -- (4);
\path (5) edge [loop below] (4);
\path (2) edge [loop above] (2);
\end{tikzpicture}
\end{frame}

\begin{frame}{Arrow Algorithm}
\centering
\begin{tikzpicture}[arvy-expl]
\node[v] (1) at (0,5) {1};
\node[q] (2) at (7,5) {2};
\node[rd, above=3pt of 2, color=white] {\footnotesize needs resource};
\node[v] (3) at (3,4) {3};
\node[v] (4) at (2,1) {4};
\node[r] (5) at (6,0) {5};
\draw[e] (3) -- (2);
\draw[e] (4) -- (3);
\draw[re] (4) --node[above right=2pt and -20pt]{\footnotesize request from 2}  (5);
\draw[e] (1) -- (4);
\path (5) edge [loop below] (4);
\path (2) edge [loop above] (2);
\end{tikzpicture}
\end{frame}

\begin{frame}{Arrow Algorithm}
\centering
\begin{tikzpicture}[arvy-expl]
\node[v] (1) at (0,5) {1};
\node[q] (2) at (7,5) {2};
\node[rd, above=3pt of 2, color=white] {\footnotesize needs resource};
\node[v] (3) at (3,4) {3};
\node[v] (4) at (2,1) {4};
\node[v] (5) at (6,0) {5};
\draw[e] (3) -- (2);
\draw[e] (4) -- (3);
\draw[e] (5) -- (4);
\draw[e] (1) -- (4);
\draw[re, gr] (5) --node[left]{\footnotesize resource} (2);
\path (5) edge [loop below, white] (4);
\path (2) edge [loop above] (2);
\end{tikzpicture}
\end{frame}

\begin{frame}{Arrow Algorithm}
\centering
\begin{tikzpicture}[arvy-expl]
\node[v] (1) at (0,5) {1};
\node[r] (2) at (7,5) {2};
\node[rd, above=3pt of 2, color=white] {\footnotesize needs resource};
\node[v] (3) at (3,4) {3};
\node[v] (4) at (2,1) {4};
\node[v] (5) at (6,0) {5};
\draw[e] (3) -- (2);
\draw[e] (4) -- (3);
\draw[e] (5) -- (4);
\draw[e] (1) -- (4);
\path (5) edge [loop below, white] (4);
\path (2) edge [loop above] (2);
\end{tikzpicture}
\end{frame}

\begin{frame}{Ivy Algorithm}
\begin{block}{Idea}
Connect to the node that sent the request, because the root will be there soon, so only a small number of hops needed to reach it in the future
\end{block}
\end{frame}

\begin{frame}{Ivy Algorithm}
\centering

\begin{tikzpicture}[arvy-expl]
\node[v] (1) at (0,5) {1};
\node[q] (2) at (7,5) {2};
\node[rd, above=3pt of 2] {\footnotesize needs resource};
\node[v] (3) at (3,4) {3};
\node[v] (4) at (2,1) {4};
\node[r] (5) at (6,0) {5};
\node[gr, above=0 of 5] {\footnotesize has resource};
\draw[e] (2) -- (3);
\draw[e] (3) -- (4);
\draw[e] (4) -- (5);
\draw[e] (1) -- (4);
\path (5) edge [loop below] (4);
\end{tikzpicture}

\end{frame}

\begin{frame}{Ivy Algorithm}
\centering

\begin{tikzpicture}[arvy-expl]
\node[v] (1) at (0,5) {1};
\node[q] (2) at (7,5) {2};
\node[rd, above=3pt of 2, color=white] {\footnotesize needs resource};
\node[v] (3) at (3,4) {3};
\node[v] (4) at (2,1) {4};
\node[r] (5) at (6,0) {5};
\draw[re] (2) --node[below right=3pt and -20pt]{\footnotesize request from 2}  (3);
\draw[e] (3) -- (4);
\draw[e] (4) -- (5);
\draw[e] (1) -- (4);
\path (5) edge [loop below] (4);
\path (2) edge [loop above] (2);
\end{tikzpicture}
\end{frame}


\begin{frame}{Ivy Algorithm}
\centering
\begin{tikzpicture}[arvy-expl]
\node[v] (1) at (0,5) {1};
\node[q] (2) at (7,5) {2};
\node[rd, above=3pt of 2, color=white] {\footnotesize needs resource};
\node[v] (3) at (3,4) {3};
\node[v] (4) at (2,1) {4};
\node[r] (5) at (6,0) {5};
\draw[e] (3) -- (2);
\draw[re] (3) --node[right]{\footnotesize request from 2}  (4);
\draw[e] (4) -- (5);
\draw[e] (1) -- (4);
\path (5) edge [loop below] (4);
\path (2) edge [loop above] (2);
\end{tikzpicture}
\end{frame}

\begin{frame}{Ivy Algorithm}
\centering
\begin{tikzpicture}[arvy-expl]
\node[v] (1) at (0,5) {1};
\node[q] (2) at (7,5) {2};
\node[rd, above=3pt of 2, color=white] {\footnotesize needs resource};
\node[v] (3) at (3,4) {3};
\node[v] (4) at (2,1) {4};
\node[r] (5) at (6,0) {5};
\draw[e] (3) -- (2);
\draw[e] (4) -- (2);
\draw[re] (4) --node[above right=2pt and -20pt]{\footnotesize request from 2}  (5);
\draw[e] (1) -- (4);
\path (5) edge [loop below] (4);
\path (2) edge [loop above] (2);
\end{tikzpicture}
\end{frame}

\begin{frame}{Ivy Algorithm}
\centering
\begin{tikzpicture}[arvy-expl]
\node[v] (1) at (0,5) {1};
\node[q] (2) at (7,5) {2};
\node[rd, above=3pt of 2, color=white] {\footnotesize needs resource};
\node[v] (3) at (3,4) {3};
\node[v] (4) at (2,1) {4};
\node[v] (5) at (6,0) {5};
\draw[e] (3) -- (2);
\draw[e] (4) -- (2);
\draw[e] (5) -- (2);
\draw[e] (1) -- (4);
\draw[re, gr] (5) to[out=100, in=240] node[left]{\footnotesize resource} (2);
\path (5) edge [loop below, white] (4);
\path (2) edge [loop above] (2);
\end{tikzpicture}
\end{frame}

\begin{frame}{Ivy Algorithm}
\centering
\begin{tikzpicture}[arvy-expl]
\node[v] (1) at (0,5) {1};
\node[r] (2) at (7,5) {2};
\node[rd, above=3pt of 2, color=white] {\footnotesize needs resource};
\node[v] (3) at (3,4) {3};
\node[v] (4) at (2,1) {4};
\node[v] (5) at (6,0) {5};
\draw[e] (3) -- (2);
\draw[e] (4) -- (2);
\draw[e] (5) -- (2);
\draw[e] (1) -- (4);
\path (5) edge [loop below, white] (4);
\path (2) edge [loop above] (2);
\end{tikzpicture}
\end{frame}

\begin{frame}{General Arvy Algorithms}
\begin{block}{Idea}
Allow connecting back to \textit{any} node on the request path
\end{block}
\end{frame}

\begin{frame}{General Arvy Algorithms}
\centering

\begin{tikzpicture}[arvy-expl]
\node[v] (1) at (0,5) {1};
\node[q] (2) at (7,5) {2};
\node[rd, above=3pt of 2] {\footnotesize needs resource};
\node[v] (3) at (3,4) {3};
\node[v] (4) at (2,1) {4};
\node[r] (5) at (6,0) {5};
\node[gr, above=0 of 5] {\footnotesize has resource};
\draw[e] (2) -- (3);
\draw[e] (3) -- (4);
\draw[e] (4) -- (5);
\draw[e] (1) -- (4);
\path (5) edge [loop below] (4);
\end{tikzpicture}

\end{frame}

\begin{frame}{General Arvy Algorithms}
\centering

\begin{tikzpicture}[arvy-expl]
\node[v] (1) at (0,5) {1};
\node[q] (2) at (7,5) {2};
\node[rd, above=3pt of 2, color=white] {\footnotesize needs resource};
\node[v] (3) at (3,4) {3};
\node[v] (4) at (2,1) {4};
\node[r] (5) at (6,0) {5};
\draw[re] (2) to[out=220,in=0] node[below right=3pt and -20pt]{\footnotesize request from 2}  (3);
\draw[cand] (3) -- (2);
\node[black!70] (cand) at (4,5.5) {\footnotesize{parent candidate}};
\draw[black!70] (cand) -- (4.9,4.6);
\draw[e] (3) -- (4);
\draw[e] (4) -- (5);
\draw[e] (1) -- (4);
\path (5) edge [loop below] (4);
\path (2) edge [loop above] (2);
\end{tikzpicture}
\end{frame}


\begin{frame}{General Arvy Algorithms}
\centering
\begin{tikzpicture}[arvy-expl]
\node[v] (1) at (0,5) {1};
\node[q] (2) at (7,5) {2};
\node[rd, above=3pt of 2, color=white] {\footnotesize needs resource};
\node[v] (3) at (3,4) {3};
\node[v] (4) at (2,1) {4};
\node[r] (5) at (6,0) {5};
\draw[e] (3) -- (2);
\draw[re] (3) to[out=-90,in=60] node[right]{\footnotesize request from 2}  (4);
\draw[cand] (4) -- (3);
\draw[cand] (4) -- (2);
\draw[e] (4) -- (5);
\draw[e] (1) -- (4);
\path (5) edge [loop below] (4);
\path (2) edge [loop above] (2);
\end{tikzpicture}
\end{frame}

\begin{frame}{General Arvy Algorithms}
\centering
\begin{tikzpicture}[arvy-expl]
\node[v] (1) at (0,5) {1};
\node[q] (2) at (7,5) {2};
\node[rd, above=3pt of 2, color=white] {\footnotesize needs resource};
\node[v] (3) at (3,4) {3};
\node[v] (4) at (2,1) {4};
\node[r] (5) at (6,0) {5};
\draw[cand] (5) -- (2);
\draw[cand] (5) -- (3);
\draw[cand] (5) -- (4);
\draw[e] (3) -- (2);
\draw[e] (4) -- (2);
\draw[re] (4) to[out=0,in=150] node[above right=2pt and -20pt]{\footnotesize request from 2}  (5);
\draw[e] (1) -- (4);
\path (5) edge [loop below] (4);
\path (2) edge [loop above] (2);
\end{tikzpicture}
\end{frame}

\begin{frame}{General Arvy Algorithms}
\centering
\begin{tikzpicture}[arvy-expl]
\node[v] (1) at (0,5) {1};
\node[q] (2) at (7,5) {2};
\node[rd, above=3pt of 2, color=white] {\footnotesize needs resource};
\node[v] (3) at (3,4) {3};
\node[v] (4) at (2,1) {4};
\node[v] (5) at (6,0) {5};
\draw[e] (3) -- (2);
\draw[e] (4) -- (2);
\draw[e] (5) -- (3);
\draw[e] (1) -- (4);
\draw[re, gr] (5) -- node[left]{\footnotesize resource} (2);
\path (5) edge [loop below, white] (4);
\path (2) edge [loop above] (2);
\end{tikzpicture}
\end{frame}

\begin{frame}{General Arvy Algorithms}
\centering
\begin{tikzpicture}[arvy-expl]
\node[v] (1) at (0,5) {1};
\node[r] (2) at (7,5) {2};
\node[rd, above=3pt of 2, color=white] {\footnotesize needs resource};
\node[v] (3) at (3,4) {3};
\node[v] (4) at (2,1) {4};
\node[v] (5) at (6,0) {5};
\draw[e] (3) -- (2);
\draw[e] (4) -- (2);
\draw[e] (5) -- (3);
\draw[e] (1) -- (4);
\path (5) edge [loop below, white] (4);
\path (2) edge [loop above] (2);
\end{tikzpicture}
\end{frame}

\begin{frame}{Edge Cost Minimizer Heuristic}
\begin{block}{Idea}
Connect to the node with minimum edge distance. This makes for a short total tree edge distance over time
\end{block}
\end{frame}

\begin{frame}{Edge Cost Minimizer Heuristic}
\centering
\begin{tikzpicture}
[arvy-expl, bn/.style={circle,draw}
,root/.style={bn,thick}
,be/.style={dashed,draw=blue!70!black,arrows={-Stealth[scale=1.5]}}
,req/.style={bn,red!70!black}
,auto,scale=1.6,font=\footnotesize]
\node[q] (n1) at (0,0) {a};
\node[bn] (n2) at (1,2) {b};
\node[bn] (n3) at (3,3) {c};
\node[bn] (n4) at (5,2) {d};
\node[bn] (n5) at (4,0) {e};
\draw[be] (n1) -- node[blue!70!black]{\footnotesize{request path}} (n2);
\draw[be] (n2) -- (n3);
\draw[be] (n3) -- (n4);
\draw[be] (n4) to[out=-100,in=50] (n5);
\draw[cand] (n5) -- node{$3$} (n1);
\draw[cand] (n5) -- node{$5$} (n2);
\draw[cand] (n5) -- node{$1$} (n3);
\draw[cand] (n5) -- node[left]{$4$} (n4);
\end{tikzpicture}

\end{frame}

\begin{frame}{Edge Cost Minimizer Heuristic}
\centering
\begin{tikzpicture}
[arvy-expl, bn/.style={circle,draw}
,root/.style={bn,thick}
,be/.style={dashed,draw=blue!70!black,arrows={-Stealth[scale=1.5]}}
,req/.style={bn,red!70!black}
,auto,scale=1.6,font=\footnotesize]
\node[q] (n1) at (0,0) {a};
\node[bn] (n2) at (1,2) {b};
\node[bn] (n3) at (3,3) {c};
\node[bn] (n4) at (5,2) {d};
\node[bn] (n5) at (4,0) {e};
\draw[be] (n1) -- node[blue!70!black]{\footnotesize{request path}} (n2);
\draw[be] (n2) -- (n3);
\draw[be] (n3) -- (n4);
\draw[cand,white] (n5) -- node[white]{$3$} (n1);
\draw[be] (n4) to[out=-100,in=50] (n5);
\draw[e] (n5) -- node{$1$} (n3);
\draw[weight] (n5) -- node{$3$} (n1);
\draw[weight] (n5) -- node{$5$} (n2);
\draw[weight] (n5) -- node[left]{$4$} (n4);
\end{tikzpicture}

\end{frame}

\begin{frame}{Local Pair Distance Minimizer Heuristic}
\centering
\begin{tikzpicture}[arvy-expl,scale=2,font=\footnotesize]
\node[q] (0) at (0,0) {a};
\node[v] (1) at (0,2) {b};
\node[v] (2) at (2,3) {c};
\node[v,white] (3) at (3,1) {d};
\draw[e] (1) --node[left]{4} (0);
\draw[cand] (2) --node[below=2pt]{6} (0);
\draw[cand] (2) --node[above]{5} (1);
\draw[re] (0) to[out=110,in=-110] node[left, blue!70!black]{\footnotesize{request path}} (1);
\draw[re] (1) to[out=60,in=180] (2);
\end{tikzpicture}
\end{frame}

\begin{frame}{Local Pair Distance Minimizer Heuristic}
\centering
\begin{tikzpicture}[arvy-expl,scale=2,font=\footnotesize]
\node[q] (0) at (0,0) {a};
\node[v] (1) at (0,2) {b};
\node[v] (2) at (2,3) {c};
\node[v,white] (3) at (3,1) {d};
\draw[e] (1) --node[left]{4} (0);
\draw[weight] (2) --node[below=2pt]{6} (0);
\draw[e] (2) --node[above]{5} (1);
\draw[re] (0) to[out=110,in=-110] node[left, blue!70!black]{\footnotesize{request path}} (1);
\draw[re] (1) to[out=60,in=180] (2);
\end{tikzpicture}
\end{frame}

\begin{frame}{Local Pair Distance Minimizer Heuristic}
\centering
\begin{tikzpicture}[arvy-expl,scale=2,font=\footnotesize]
\node[q] (0) at (0,0) {a};
\node[v] (1) at (0,2) {b};
\node[v] (2) at (2,3) {c};
\node[v] (3) at (3,1) {d};
\draw[e] (1) --node[left]{4} (0);
\draw[weight] (2) --node[below=2pt]{6} (0);
\draw[e] (2) --node[above]{5} (1);
\draw[cand] (3) --node[below]{2} (0);
\draw[cand] (3) --node[above]{3} (1);
\draw[cand] (3) --node[right]{2} (2);
\draw[re] (0) to[out=110,in=-110] node[left, blue!70!black]{\footnotesize{request path}} (1);
\draw[re] (1) to[out=60,in=180] (2);
\draw[re] (2) to[out=-30,in=90] (3);
\end{tikzpicture}
\end{frame}

\begin{frame}{Local Pair Distance Minimizer Heuristic}
\centering
\begin{tikzpicture}[arvy-expl,scale=2,font=\footnotesize]
\node[q] (0) at (0,0) {a};
\node[v] (1) at (0,2) {b};
\node[v] (2) at (2,3) {c};
\node[v] (3) at (3,1) {d};
\draw[e] (1) --node[left]{4} (0);
\draw[weight] (2) --node[below=2pt]{6} (0);
\draw[e] (2) --node[above]{5} (1);
\draw[weight] (0) --node[below]{2} (3);
\draw[e] (3) --node[above]{3} (1);
\draw[weight] (2) --node[right]{2} (3);
\draw[re] (0) to[out=110,in=-110] node[left, blue!70!black]{\footnotesize{request path}} (1);
\draw[re] (1) to[out=60,in=180] (2);
\draw[re] (2) to[out=-30,in=90] (3);
\end{tikzpicture}
\end{frame}

\begin{frame}{Dynamic Star Heuristic}
\begin{block}{Idea}
\begin{itemize}
\item Measure request probabilities $p_v$ for each node $v$
\item Create a star with the best center node over time
\end{itemize}
\end{block}
\end{frame}

\begin{frame}{Results}
\begin{itemize}
\item Performance: Total time needed to satisfy all requests
\item Graph costs: Uniformly random nodes in a unit square
\item Request sequence: Uniformly random or adversarial
\end{itemize}
\end{frame}

\newcommand{\evalTime}{Average request time}
\newcommand{\evalEdges}{Average tree edge distance}

\pgfplotsset{
  every axis legend/.append style={
    at={(0.5,1.03)},
    anchor=south
  },
  legend cell align = left,
  legend image post style = {
    line width = 2pt,
  },
  width=0.9\textwidth,
  height=0.5\textwidth,
  legend columns = 3,
  every axis/.append style = {
    xmode = log,
    xmin = 1,
    grid,
    colormap name = colormap/jet,
    legend style = { font = \scriptsize },
    xlabel = Number of requests,
    no markers,
  },
  every axis plot post/.append style = {
    thick,
    x={x},
    line join=round,
  },
}

\begin{frame}{Results: Best Tree for Arrow}

\begin{tikzpicture}
\centering
\begin{axis}[
  ylabel = \evalTime,
  xmax = 100000,
  cycle list = { [samples of colormap=5] },
  legend columns = 2,
  ymin = 1,
  ymax = 3,
  font=\footnotesize,
]
\pgfplotstableread{data/trees/time.dat}{\data}
\addplot table [y={arrow-random}] {\data};
\addlegendentry {Random tree}
\addplot table [y={arrow-mst}] {\data};
\addlegendentry {Minimum spanning tree}
\addplot table [y={arrow-shortpairs}] {\data};
\addlegendentry {\hyperref[tree:ampd]{Approx. min. pair dist.} tree}
\addplot table [y={arrow-star}] {\data};
\addlegendentry {\hyperref[tree:star]{Best star} tree}
\addplot table [y={arrow-shortestpairs}] {\data};
\addlegendentry {\hyperref[tree:mpd]{Min. pair dist.} tree}
\end{axis}
\end{tikzpicture}
\end{frame}

\begin{frame}{Results: Tree Convergence}


\begin{tikzpicture}
\centering
\begin{axis}[
  ylabel = \evalEdges,
  xmax = 100000,
  cycle list = { [samples of colormap=8] },
  font=\footnotesize
]
\pgfplotstableread{data/converging/treeWeight.dat}{\data}
\addplot+[dashed] table [y={arrow-random}] {\data};
\addlegendentry {Uniformly random tree}
\addplot+[dashed] table [y={arrow-star}] {\data};
\addlegendentry {Best star tree}
\addplot+[dashed] table [y={arrow-mst}] {\data};
\addlegendentry {Min. sp. tree}
\addplot table [y={random-random}] {\data};
\addlegendentry {Random heuristic}
\addplot table [y={ivy-random}] {\data};
\addlegendentry {Ivy}
\addplot table [y={dynamicStar-random}] {\data};
\addlegendentry {\hyperref[alg:dynstar]{Dynamic Star}}
\addplot table [y={localMinPairs-random}] {\data};
\addlegendentry {\hyperref[alg:lpm]{Local Pair Dist. Min.}}
\addplot table [y={minWeight-random}] {\data};
\addlegendentry {\hyperref[alg:ecm]{Edge cost min.}}
\end{axis}
\end{tikzpicture}

\end{frame}

\begin{frame}{Results: Best Heuristic for Random Requests}

\begin{tikzpicture}
\centering
\begin{axis}[
  ylabel = \evalTime,
  xmax = 1000000,
  cycle list = { [samples of colormap=3] },
  legend columns = 2,
  font=\footnotesize,
]
\pgfplotstableread{data/algs/time.dat}{\data}
\addplot table [y={arrow-star}] {\data};
\addlegendentry {Best star Arrow}
\addplot table [y={ivy-random}] {\data};
\addlegendentry {Ivy}
\addplot table [y={localMinPairs-random}] {\data};
\addlegendentry {\hyperref[alg:lpm]{Local Pair Dist. Min.}}

\end{axis}
\end{tikzpicture}
\end{frame}

\begin{frame}{Results: Best Heuristic for Random Requests}

\begin{tikzpicture}
\begin{axis}[
  xmax = 1000000,
  cycle list = { [samples of colormap=5] },
  legend columns = 2,
  ylabel=\evalTime,
  font=\footnotesize,
]
\pgfplotstableread{data/adversary/time.dat}{\data}
\addplot table [y={arrow-star}] {\data};
\addlegendentry {Best star Arrow}
\addplot table [y={ivy-random}] {\data};
\addlegendentry {Ivy}
\addplot table [y={localMinPairs-random}] {\data};
\addlegendentry {\hyperref[alg:lpm]{Local Pair Dist. Min.}}
\addplot table [y={dynamicStar-random}] {\data};
\addlegendentry {\hyperref[alg:dynstar]{Dynamic Star}}

\end{axis}
\end{tikzpicture}

\end{frame}

\begin{frame}{Ivy in Small Cliques}
We compare the average time to satisfy a request between Arrow and Ivy in small cliques:
\begin{center}
\npdecimalsign{.}
\nprounddigits{3}
\begin{tabular}{ r | n{1}{3} | n{1}{3} }
  Node count & Arrow & Ivy \\
  \hline
  3 & 1.334 & {\color{ForestGreen} 1.250} \\
  \hline
  4 & 1.498 & {\color{ForestGreen} 1.443} \\
  \hline
  5 & {\color{ForestGreen} 1.599} & 1.603 \\
  \hline
  6 & {\color{ForestGreen} 1.664} & 1.738 \\
  \hline
  7 & {\color{ForestGreen} 1.712} & 1.858 \\
  \hline
  8 & {\color{ForestGreen} 1.749} & 1.963 \\
  \hline
  $\vdots$ & {\color{ForestGreen} $\vdots$} & $\vdots$ \\
\end{tabular}
\end{center}

\end{frame}

\begin{frame}{Conclusion}

\begin{itemize}
\item Arrow with a star is very good for uniformly random requests
\item The Dynamic Star heuristic outperforms all others for adversarial requests
\item The Local Pair Distance Minimizer heuristic is worth looking into for less node contention
\end{itemize}

\end{frame}

\end{document}
