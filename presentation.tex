\documentclass[12pt,hyperref={pdfpagelabels=false}]{beamer}
\let\Tiny=\tiny

\usepackage{xspace}
\usepackage{xmpmulti}
\usepackage{array}
\usepackage[absolute,overlay]{textpos}
\usepackage{forloop}
\usepackage{pgfplots}
\usepackage{subcaption}
\usepackage{tikz}

\usetikzlibrary{positioning}
\usetikzlibrary{arrows.meta}

%\usecolortheme{rose}

\setbeamerfont{frametitle}{series=\bfseries}
\setbeamercolor*{title}{fg=white}
\setbeamercolor*{author}{fg=white}
\setbeamercolor*{institute}{fg=white}
\setbeamercolor*{date}{fg=white}
\setbeamerfont{title}{series=\bfseries,size=\huge}
% \setbeamercolor{frametitle}{}
\setbeamerfont{author}{series=\bfseries}


\usetheme{Pittsburgh}
\setbeamertemplate{navigation symbols}{}

\title{Arvy Heuristics for Distributed Mutual Exclusion}
\author{Silvan Mosberger}
\institute{ETH Zurich -- Distributed Computing Group -- www.disco.ethz.ch}

\begin{document}

{
\usebackgroundtemplate{\includegraphics[width=\paperwidth]{figures/bg}}
\begin{frame}
	\begin{textblock*}{\paperwidth}[0,0](0cm,1cm)
		\begin{center}
			\usebeamercolor[fg]{title}
			\textbf{\huge \inserttitle}
		\end{center}
		\usebeamercolor[fg]{normal text}
	\end{textblock*}
	\begin{textblock*}{\paperwidth}[0,0](-0.5cm,7.4cm)
		\flushright
		\color{white}
		\itshape \insertauthor
		\usebeamercolor[fg]{normal text}
	\end{textblock*}
	\begin{textblock*}{\paperwidth}[0,1](0.2cm,9.4cm)
		\flushleft
 		\usebeamercolor[fg]{institute}
		\tiny \itshape \insertinstitute
		\usebeamercolor[fg]{normal text}
	\end{textblock*}
\end{frame}
}

% Explain the problem

\begin{frame}{Distributed Mutual Exclusion}

The guarantee that only at most one node in a network can have exclusive access to a shared resource

\begin{block}{Examples}
\begin{itemize}
\item A distributed counter where every node can increase it
\item Shared memory in a multiprocessor machine
\end{itemize}

\end{block}

\end{frame}

\tikzset{
  arvy-expl/.style =
    { v/.style = { circle, draw } % normal graph vertices
    , r/.style = { v, gr } % root nodes
    , rd/.style = { red!70!black }
    , bl/.style = { blue!70!black }
    , gr/.style = { green!50!black }
    , q/.style = { v, rd } % currently making request
    , e/.style = { draw=black, -> }
    , re/.style = { e, dashed, bl }
    , cand/.style = { dotted, gray, ->, >={Stealth[scale=1]} }
    , >={Stealth[scale=2]}
    , scale = 1.0
    , auto
    }
}

\begin{frame}{Arrow}
\begin{block}{Idea}
Maintain a rooted spanning tree with every node (transitively) pointing towards the node holding the token.
\end{block}
\end{frame}

\begin{frame}{Arrow}
\centering

\begin{tikzpicture}[arvy-expl]
\node[v] (1) at (0,5) {1};
\node[q] (2) at (7,5) {2};
\node[rd, above=3pt of 2] {\footnotesize needs resource};
\node[v] (3) at (3,4) {3};
\node[v] (4) at (2,1) {4};
\node[r] (5) at (6,0) {5};
\node[gr, above=0 of 5] {\footnotesize has resource};
\draw[e] (2) -- (3);
\draw[e] (3) -- (4);
\draw[e] (4) -- (5);
\draw[e] (1) -- (4);
\path (5) edge [loop below] (4);
\end{tikzpicture}

\end{frame}

\begin{frame}{Arrow}
\centering

\begin{tikzpicture}[arvy-expl]
\node[v] (1) at (0,5) {1};
\node[q] (2) at (7,5) {2};
\node[rd, above=3pt of 2, color=white] {\footnotesize needs resource};
\node[v] (3) at (3,4) {3};
\node[v] (4) at (2,1) {4};
\node[r] (5) at (6,0) {5};
\draw[re] (2) --node[below right=3pt and -20pt]{\footnotesize request from 2}  (3);
\draw[e] (3) -- (4);
\draw[e] (4) -- (5);
\draw[e] (1) -- (4);
\path (5) edge [loop below] (4);
\path (2) edge [loop above] (2);
\end{tikzpicture}
\end{frame}

\begin{frame}{Arrow}
\centering
\begin{tikzpicture}[arvy-expl]
\node[v] (1) at (0,5) {1};
\node[q] (2) at (7,5) {2};
\node[rd, above=3pt of 2, color=white] {\footnotesize needs resource};
\node[v] (3) at (3,4) {3};
\node[v] (4) at (2,1) {4};
\node[r] (5) at (6,0) {5};
\draw[e] (3) -- (2);
\draw[re] (3) --node[right]{\footnotesize request from 2}  (4);
\draw[e] (4) -- (5);
\draw[e] (1) -- (4);
\path (5) edge [loop below] (4);
\path (2) edge [loop above] (2);
\end{tikzpicture}
\end{frame}

\begin{frame}{Arrow}
\centering
\begin{tikzpicture}[arvy-expl]
\node[v] (1) at (0,5) {1};
\node[q] (2) at (7,5) {2};
\node[rd, above=3pt of 2, color=white] {\footnotesize needs resource};
\node[v] (3) at (3,4) {3};
\node[v] (4) at (2,1) {4};
\node[r] (5) at (6,0) {5};
\draw[e] (3) -- (2);
\draw[e] (4) -- (3);
\draw[re] (4) --node[above right=2pt and -20pt]{\footnotesize request from 2}  (5);
\draw[e] (1) -- (4);
\path (5) edge [loop below] (4);
\path (2) edge [loop above] (2);
\end{tikzpicture}
\end{frame}

\begin{frame}{Arrow}
\centering
\begin{tikzpicture}[arvy-expl]
\node[v] (1) at (0,5) {1};
\node[q] (2) at (7,5) {2};
\node[rd, above=3pt of 2, color=white] {\footnotesize needs resource};
\node[v] (3) at (3,4) {3};
\node[v] (4) at (2,1) {4};
\node[v] (5) at (6,0) {5};
\draw[e] (3) -- (2);
\draw[e] (4) -- (3);
\draw[e] (5) -- (4);
\draw[e] (1) -- (4);
\draw[re, gr] (5) --node[left]{\footnotesize resource} (2);
\path (5) edge [loop below, white] (4);
\path (2) edge [loop above] (2);
\end{tikzpicture}
\end{frame}

\begin{frame}{Arrow}
\centering
\begin{tikzpicture}[arvy-expl]
\node[v] (1) at (0,5) {1};
\node[r] (2) at (7,5) {2};
\node[rd, above=3pt of 2, color=white] {\footnotesize needs resource};
\node[v] (3) at (3,4) {3};
\node[v] (4) at (2,1) {4};
\node[v] (5) at (6,0) {5};
\draw[e] (3) -- (2);
\draw[e] (4) -- (3);
\draw[e] (5) -- (4);
\draw[e] (1) -- (4);
\path (5) edge [loop below, white] (4);
\path (2) edge [loop above] (2);
\end{tikzpicture}
\end{frame}

\begin{frame}{Ivy}
\begin{block}{Idea}
Connect to the node that sent the request, because the root will be there soon, so only a small number of hops needed to reach it in the future
\end{block}
\end{frame}

\begin{frame}{Ivy}
\centering

\begin{tikzpicture}[arvy-expl]
\node[v] (1) at (0,5) {1};
\node[q] (2) at (7,5) {2};
\node[rd, above=3pt of 2] {\footnotesize needs resource};
\node[v] (3) at (3,4) {3};
\node[v] (4) at (2,1) {4};
\node[r] (5) at (6,0) {5};
\node[gr, above=0 of 5] {\footnotesize has resource};
\draw[e] (2) -- (3);
\draw[e] (3) -- (4);
\draw[e] (4) -- (5);
\draw[e] (1) -- (4);
\path (5) edge [loop below] (4);
\end{tikzpicture}

\end{frame}

\begin{frame}{Ivy}
\centering

\begin{tikzpicture}[arvy-expl]
\node[v] (1) at (0,5) {1};
\node[q] (2) at (7,5) {2};
\node[rd, above=3pt of 2, color=white] {\footnotesize needs resource};
\node[v] (3) at (3,4) {3};
\node[v] (4) at (2,1) {4};
\node[r] (5) at (6,0) {5};
\draw[re] (2) --node[below right=3pt and -20pt]{\footnotesize request from 2}  (3);
\draw[e] (3) -- (4);
\draw[e] (4) -- (5);
\draw[e] (1) -- (4);
\path (5) edge [loop below] (4);
\path (2) edge [loop above] (2);
\end{tikzpicture}
\end{frame}


\begin{frame}{Ivy}
\centering
\begin{tikzpicture}[arvy-expl]
\node[v] (1) at (0,5) {1};
\node[q] (2) at (7,5) {2};
\node[rd, above=3pt of 2, color=white] {\footnotesize needs resource};
\node[v] (3) at (3,4) {3};
\node[v] (4) at (2,1) {4};
\node[r] (5) at (6,0) {5};
\draw[e] (3) -- (2);
\draw[re] (3) --node[right]{\footnotesize request from 2}  (4);
\draw[e] (4) -- (5);
\draw[e] (1) -- (4);
\path (5) edge [loop below] (4);
\path (2) edge [loop above] (2);
\end{tikzpicture}
\end{frame}

\begin{frame}{Ivy}
\centering
\begin{tikzpicture}[arvy-expl]
\node[v] (1) at (0,5) {1};
\node[q] (2) at (7,5) {2};
\node[rd, above=3pt of 2, color=white] {\footnotesize needs resource};
\node[v] (3) at (3,4) {3};
\node[v] (4) at (2,1) {4};
\node[r] (5) at (6,0) {5};
\draw[e] (3) -- (2);
\draw[e] (4) -- (2);
\draw[re] (4) --node[above right=2pt and -20pt]{\footnotesize request from 2}  (5);
\draw[e] (1) -- (4);
\path (5) edge [loop below] (4);
\path (2) edge [loop above] (2);
\end{tikzpicture}
\end{frame}

\begin{frame}{Ivy}
\centering
\begin{tikzpicture}[arvy-expl]
\node[v] (1) at (0,5) {1};
\node[q] (2) at (7,5) {2};
\node[rd, above=3pt of 2, color=white] {\footnotesize needs resource};
\node[v] (3) at (3,4) {3};
\node[v] (4) at (2,1) {4};
\node[v] (5) at (6,0) {5};
\draw[e] (3) -- (2);
\draw[e] (4) -- (2);
\draw[e] (5) -- (2);
\draw[e] (1) -- (4);
\draw[re, gr] (5) to[out=100, in=240] node[left]{\footnotesize resource} (2);
\path (5) edge [loop below, white] (4);
\path (2) edge [loop above] (2);
\end{tikzpicture}
\end{frame}

\begin{frame}{Ivy}
\centering
\begin{tikzpicture}[arvy-expl]
\node[v] (1) at (0,5) {1};
\node[r] (2) at (7,5) {2};
\node[rd, above=3pt of 2, color=white] {\footnotesize needs resource};
\node[v] (3) at (3,4) {3};
\node[v] (4) at (2,1) {4};
\node[v] (5) at (6,0) {5};
\draw[e] (3) -- (2);
\draw[e] (4) -- (2);
\draw[e] (5) -- (2);
\draw[e] (1) -- (4);
\path (5) edge [loop below, white] (4);
\path (2) edge [loop above] (2);
\end{tikzpicture}
\end{frame}

\begin{frame}{Arvy}
\begin{block}{Idea}
Allow connecting back to \textit{any} node on the request path
\end{block}
\end{frame}

\begin{frame}{Arvy}
\centering

\begin{tikzpicture}[arvy-expl]
\node[v] (1) at (0,5) {1};
\node[q] (2) at (7,5) {2};
\node[rd, above=3pt of 2] {\footnotesize needs resource};
\node[v] (3) at (3,4) {3};
\node[v] (4) at (2,1) {4};
\node[r] (5) at (6,0) {5};
\node[gr, above=0 of 5] {\footnotesize has resource};
\draw[e] (2) -- (3);
\draw[e] (3) -- (4);
\draw[e] (4) -- (5);
\draw[e] (1) -- (4);
\path (5) edge [loop below] (4);
\end{tikzpicture}

\end{frame}

\begin{frame}{Arvy}
\centering

\begin{tikzpicture}[arvy-expl]
\node[v] (1) at (0,5) {1};
\node[q] (2) at (7,5) {2};
\node[rd, above=3pt of 2, color=white] {\footnotesize needs resource};
\node[v] (3) at (3,4) {3};
\node[v] (4) at (2,1) {4};
\node[r] (5) at (6,0) {5};
\draw[re] (2) to[out=220,in=0] node[below right=3pt and -20pt]{\footnotesize request from 2}  (3);
\draw[cand] (3) -- (2);
\node[black!70] (cand) at (4,5.5) {\footnotesize{parent candidate}};
\draw[black!70] (cand) -- (4.9,4.6);
\draw[e] (3) -- (4);
\draw[e] (4) -- (5);
\draw[e] (1) -- (4);
\path (5) edge [loop below] (4);
\path (2) edge [loop above] (2);
\end{tikzpicture}
\end{frame}


\begin{frame}{Arvy}
\centering
\begin{tikzpicture}[arvy-expl]
\node[v] (1) at (0,5) {1};
\node[q] (2) at (7,5) {2};
\node[rd, above=3pt of 2, color=white] {\footnotesize needs resource};
\node[v] (3) at (3,4) {3};
\node[v] (4) at (2,1) {4};
\node[r] (5) at (6,0) {5};
\draw[e] (3) -- (2);
\draw[re] (3) to[out=-90,in=60] node[right]{\footnotesize request from 2}  (4);
\draw[cand] (4) -- (3);
\draw[cand] (4) -- (2);
\draw[e] (4) -- (5);
\draw[e] (1) -- (4);
\path (5) edge [loop below] (4);
\path (2) edge [loop above] (2);
\end{tikzpicture}
\end{frame}

\begin{frame}{Arvy}
\centering
\begin{tikzpicture}[arvy-expl]
\node[v] (1) at (0,5) {1};
\node[q] (2) at (7,5) {2};
\node[rd, above=3pt of 2, color=white] {\footnotesize needs resource};
\node[v] (3) at (3,4) {3};
\node[v] (4) at (2,1) {4};
\node[r] (5) at (6,0) {5};
\draw[cand] (5) -- (2);
\draw[cand] (5) -- (3);
\draw[cand] (5) -- (4);
\draw[e] (3) -- (2);
\draw[e] (4) -- (2);
\draw[re] (4) to[out=0,in=150] node[above right=2pt and -20pt]{\footnotesize request from 2}  (5);
\draw[e] (1) -- (4);
\path (5) edge [loop below] (4);
\path (2) edge [loop above] (2);
\end{tikzpicture}
\end{frame}

\begin{frame}{Arvy}
\centering
\begin{tikzpicture}[arvy-expl]
\node[v] (1) at (0,5) {1};
\node[q] (2) at (7,5) {2};
\node[rd, above=3pt of 2, color=white] {\footnotesize needs resource};
\node[v] (3) at (3,4) {3};
\node[v] (4) at (2,1) {4};
\node[v] (5) at (6,0) {5};
\draw[e] (3) -- (2);
\draw[e] (4) -- (2);
\draw[e] (5) -- (3);
\draw[e] (1) -- (4);
\draw[re, gr] (5) -- node[left]{\footnotesize resource} (2);
\path (5) edge [loop below, white] (4);
\path (2) edge [loop above] (2);
\end{tikzpicture}
\end{frame}

\begin{frame}{Arvy}
\centering
\begin{tikzpicture}[arvy-expl]
\node[v] (1) at (0,5) {1};
\node[r] (2) at (7,5) {2};
\node[rd, above=3pt of 2, color=white] {\footnotesize needs resource};
\node[v] (3) at (3,4) {3};
\node[v] (4) at (2,1) {4};
\node[v] (5) at (6,0) {5};
\draw[e] (3) -- (2);
\draw[e] (4) -- (2);
\draw[e] (5) -- (3);
\draw[e] (1) -- (4);
\path (5) edge [loop below, white] (4);
\path (2) edge [loop above] (2);
\end{tikzpicture}
\end{frame}

\begin{frame}{Edge Cost Minimizer}
\begin{block}{Idea}
Connect to the node with minimum edge distance for tree edges as short as possible
\end{block}
\end{frame}

\begin{frame}{Edge Cost Minimizer}
\centering
\begin{tikzpicture}
[arvy-expl, bn/.style={circle,draw}
,root/.style={bn,thick}
,be/.style={dashed,draw=blue!70!black,arrows={-Stealth[scale=1.5]}}
,req/.style={bn,red!70!black}
,auto,scale=1.6]
\node[bn] (n1) at (0,0) {a};
\node[bn] (n2) at (1,2) {b};
\node[bn] (n3) at (3,3) {c};
\node[bn] (n4) at (5,2) {d};
\node[bn] (n5) at (4,0) {e};
\draw[be] (n1) -- node[blue!70!black]{\footnotesize{request path}} (n2);
\draw[be] (n2) -- (n3);
\draw[be] (n3) -- (n4);
\draw[be] (n4) to[out=-100,in=50] (n5);
\draw[cand] (n5) -- node{$3$} (n1);
\draw[cand] (n5) -- node{$5$} (n2);
\draw[cand] (n5) -- node{$1$} (n3);
\draw[cand] (n5) -- node[left]{$4$} (n4);
\end{tikzpicture}

\end{frame}

\begin{frame}{Edge Cost Minimizer}
\centering
\begin{tikzpicture}
[arvy-expl, bn/.style={circle,draw}
,root/.style={bn,thick}
,be/.style={dashed,draw=blue!70!black,arrows={-Stealth[scale=1.5]}}
,req/.style={bn,red!70!black}
,auto,scale=1.6]
\node[bn] (n1) at (0,0) {a};
\node[bn] (n2) at (1,2) {b};
\node[bn] (n3) at (3,3) {c};
\node[bn] (n4) at (5,2) {d};
\node[bn] (n5) at (4,0) {e};
\draw[be] (n1) -- node[blue!70!black]{\footnotesize{request path}} (n2);
\draw[be] (n2) -- (n3);
\draw[be] (n3) -- (n4);
\draw[cand,white] (n5) -- node[white]{$3$} (n1);
\draw[be] (n4) to[out=-100,in=50] (n5);
\draw[e] (n5) -- node{$1$} (n3);
\end{tikzpicture}

\end{frame}


\end{document}
